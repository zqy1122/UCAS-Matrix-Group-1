% XeLatex编辑
\documentclass[10pt, a4paper]{article}

\usepackage{ucas_matrix}

\begin{document}

\ucascover

% --- 目录 ---
% 会自动生成带“目录”标题的目录页
\tableofcontents
\newpage

% --- 正文开始 ---

\section{一级标题 (Section)}
这是一级标题,代表文档中最大、最重要的部分,例如书本中的“章”。段落内容直接写在这里即可。段落与段落之间会自动添加垂直间距。

\subsection{二级标题 (Subsection)}
这是二级标题,用于组织一级标题内部的各个小节。

\subsubsection{三级标题 (Subsubsection)}
这是三级标题,用于更细致地划分内容。这是模板支持的最低级别的自动编号标题。

\section{模板功能演示}
本节将展示此模板中预设的各种排版元素的使用方法。

\section{线性代数公式示例}
本节展示了在线性代数类论文或教材中常用的公式书写方式,可作为插入数学表达的参考。

\subsection{基本符号与矩阵表示}
一个 $m\times n$ 的矩阵通常记为
\[
\mathbf{A} = 
\begin{bmatrix}
a_{11} & a_{12} & \cdots & a_{1n} \\
a_{21} & a_{22} & \cdots & a_{2n} \\
\vdots & \vdots & \ddots & \vdots \\
a_{m1} & a_{m2} & \cdots & a_{mn}
\end{bmatrix}.
\]
列向量和行向量分别表示为
\[
\mathbf{x} = 
\begin{bmatrix}
x_1 \\ x_2 \\ \vdots \\ x_n
\end{bmatrix},
\quad
\mathbf{x}^\mathsf{T} = [x_1, x_2, \dots, x_n].
\]

\subsection{矩阵运算}
矩阵的加法与数乘满足:
\[
\mathbf{A} + \mathbf{B} = (a_{ij} + b_{ij}), \qquad
c\mathbf{A} = (c a_{ij}).
\]
矩阵乘法定义为:
\[
(\mathbf{A}\mathbf{B})_{ij} = \sum_{k=1}^{n} a_{ik}b_{kj}.
\]
矩阵转置、逆矩阵与迹的常见表示:
\[
\mathbf{A}^\mathsf{T}, \quad \mathbf{A}^{-1}, \quad \mathrm{tr}(\mathbf{A}) = \sum_{i} a_{ii}.
\]

\subsection{行列式与性质}
二维和三维矩阵的行列式分别为:
\[
\det
\begin{pmatrix}
a & b \\
c & d
\end{pmatrix}
= ad - bc,
\]
\[
\det
\begin{pmatrix}
a & b & c\\
d & e & f\\
g & h & i
\end{pmatrix}
= aei + bfg + cdh - ceg - bdi - afh.
\]

行列式的性质:
\[
\det(\mathbf{A}\mathbf{B}) = \det(\mathbf{A})\det(\mathbf{B}), \qquad
\det(\mathbf{A}^\mathsf{T}) = \det(\mathbf{A}).
\]

\subsection{特征值与特征向量}
如果 $\mathbf{A}\mathbf{x} = \lambda \mathbf{x}$,则 $\lambda$ 称为 $\mathbf{A}$ 的特征值,$\mathbf{x}$ 为对应的特征向量。  
特征方程为:
\[
\det(\mathbf{A} - \lambda \mathbf{I}) = 0.
\]

\subsection{正交与投影}
向量 $\mathbf{u}, \mathbf{v}$ 正交当且仅当:
\[
\mathbf{u}^\mathsf{T} \mathbf{v} = 0.
\]
$\mathbf{v}$ 在 $\mathbf{u}$ 上的投影为:
\[
\mathrm{proj}_{\mathbf{u}}(\mathbf{v}) = 
\frac{\mathbf{u}^\mathsf{T}\mathbf{v}}{\mathbf{u}^\mathsf{T}\mathbf{u}} \mathbf{u}.
\]

\subsection{奇异值分解 (SVD)}
任意实矩阵 $\mathbf{A}\in\mathbb{R}^{m\times n}$ 可分解为:
\[
\mathbf{A} = \mathbf{U}\boldsymbol{\Sigma}\mathbf{V}^\mathsf{T},
\]
其中 $\mathbf{U}\in\mathbb{R}^{m\times m}$、$\mathbf{V}\in\mathbb{R}^{n\times n}$ 为正交矩阵,
$\boldsymbol{\Sigma}$ 为仅含非负元素的对角矩阵:
\[
\boldsymbol{\Sigma} =
\begin{bmatrix}
\sigma_1 &  &  & 0\\
 & \sigma_2 &  & \\
 &  & \ddots & \\
0 &  &  & \sigma_r
\end{bmatrix}.
\]

\subsection{线性方程组与矩阵形式}
线性方程组
\[
\begin{cases}
a_{11}x_1 + a_{12}x_2 + \cdots + a_{1n}x_n = b_1,\\
a_{21}x_1 + a_{22}x_2 + \cdots + a_{2n}x_n = b_2,\\
\vdots \\
a_{m1}x_1 + a_{m2}x_2 + \cdots + a_{mn}x_n = b_m
\end{cases}
\]
可简写为矩阵形式:
\[
\mathbf{A}\mathbf{x} = \mathbf{b}.
\]
若 $\mathbf{A}$ 可逆,则解为:
\[
\mathbf{x} = \mathbf{A}^{-1}\mathbf{b}.
\]


\subsection{插入图片}
UCAS校徽

\begin{figure}[!htbp]
    \centering
    \includegraphics[width =.4\textwidth]{images/ucas_logo.pdf}
    \caption{中国科学院大学}
    \label{UCAS}
\end{figure}

\subsection{自定义环境演示}

\subsubsection{需要强调的定理}
\begin{bluebox}{解集合的三种可能性}
\begin{itemize}[leftmargin=*, labelsep=0.5em, itemsep=0.5em, topsep=0.5em]
  \item[■] \textbf{唯一解}:存在一组且仅有一组 $x_{i}$ 的值,同时满足所有方程
  \item[■] \textbf{无解}:不存在一组 $x_{i}$ 的值,同时满足所有方程—解集为空
  \item[■] \textbf{无限多解}:存在无穷多组不同的 $x_{i}$ 值,同时满足所有方程。很容易证明,如果一个系统有多个解,那么它就有无穷多个解。例如,一个系统不可能有确切的两个不同的解。
\end{itemize}
\end{bluebox}

\subsubsection{需要展示的例子}
% 使用方法:\begin{example}[可选的标题] ... \end{example}
% 注意:标题在方括号 [] 中

\begin{example}{Gauss-Jordan elimination}
首先将方程组转换成增广矩阵形式
$$
\left\{\begin{array}{r}
x_{2}-x_{3}=3 \\
-2 x_{1}+4 x_{2}-x_{3}=1 \\
-2 x_{1}+5 x_{2}-4 x_{3}=-2
\end{array} \Longrightarrow\left(\begin{array}{rrr|r}
0 & 1 & -1 & 3 \\
-2 & 4 & -1 & 1 \\
-2 & 5 & -4 & -2
\end{array}\right)\right.
$$
由于第一个主元不能为零,所以我们将不得不进行行交换,以将一个非零元素带到主元位置。
\end{example}

\subsubsection{需要展示的作业题}
\begin{exercise}{求解线性方程组}
Use Gaussian elimination with back substitution to solve the following system:
$$
\begin{aligned}
& x_{1}+x_{2}+x_{3}=1, \\
& x_{1}+2 x_{2}+2 x_{3}=1, \\
& x_{1}+2 x_{2}+3 x_{3}=1 .
\end{aligned}
$$
\end{exercise}


\section{写在最后}
\subsection{开源地址}
\begin{itemize}
    \item Github: \url{https://github.com/neverwinHao/UCAS-Matrix-Template}
\end{itemize}
\end{document}